\documentclass[11pt,a4paper,oneside]{article}

\usepackage[T2A]{fontenc}
\usepackage[utf8]{inputenc}
\usepackage[english, russian]{babel}
\usepackage{graphicx}
\usepackage{amsmath}
\usepackage{amssymb}
\usepackage{color} 
\usepackage{import} 
\usepackage{epigraph}
\usepackage{daytime} 
\usepackage{wrapfig}
\usepackage{verbatim}
\usepackage[russian]{hyperref}
\usepackage{listings}






\begin{document}

\renewcommand{\t}{\texttt}
\newcommand{\giturl}[1]{\href{https://github.com/GassaFM/olymp.sty}{#1}}
\newcommand{\subgiturl}[2]{\href{https://github.com/GassaFM/olymp.sty/#1}{#2}}
\newcommand{\command}[1]{\t{\textbackslash#1}}

\title{Руководство пользователя пакета olymp.sty}
\author{}
\maketitle
\newpage

\tableofcontents
\newpage

\section{Введение}

Пакет \t{olymp.sty} предназначен для подготовки условий задач
соревнований по программированию. 

На текущий момент(\today) исходный код библиотеки, а также
этой документации, хранится в \giturl{репозитории} на GitHub.

\newpage
\section{Создание условия одной задачи}

Для условия одной задачи используется окружение \t{problem}.

У этого окружения есть следующие параметры:
\begin{enumerate}
\item Название задачи
\item Имя входного файла
\item Имя выходного файла
\item Ограничение по времени
\item Ограничение по памяти
\item Отображение результатов (может быть опущен)
\end{enumerate}

Например, началом задачи может быть следующая команда:
\begin{lstlisting}[language=tex]
\begin{problem}{Name}{file.in}{file.out}{2 seconds}{256 mebibytes}
\end{lstlisting}
Такая команда создаст шапку задачи с названием и ограничениями.

Внутри окружения \t{problem} определены секции задачи, приведённые в таблице ниже.
Секции являются заголовками часто встречающихся частей задачи.

\begin{tabular}{|c|c|c|}
\hline
Назавние секции & Текст в русском условии & Текст в английском условии \\ \hline
\command{Constraints} & Ограничения & Constraints \\ \hline
\command{Example} & Пример & Example \\ \hline
\command{Examples} & Примеры & Examples \\ \hline
\command{Explanation} & Пояснение к примеру & Explanation \\ \hline
\command{Explanations} & Пояснения к примерам & Explanations \\ \hline
\command{Illustration} & Иллюстрация & Illustration \\ \hline
\command{InputFile} & Формат входных данных & Input \\ \hline
\command{Note} & Замечание & Note \\ \hline
\command{Notes} & Замечания & Notes \\ \hline
\command{OutputFile} & Формат выходных данных & Output \\ \hline
\command{Scoring} & Система оценки & Scoring \\ \hline
\command{Specification} & Спецификация & Specification \\ \hline
\command{Subtask} & Подзадача \#1 & Subtask \#1\\ \hline
\command{SubtaskWithCost} & Подзадача \#1 (\#2 баллов) & Subtask \#1 (\#2 points)\\ \hline
\end{tabular}

У последних двух секций есть аргументы.
Их значения представлены как \#1 и \#2 соответсвенно.

Кроме того, в окружении \t{problem} определены окружения примеров \t{example}, \t{examplewide} и \t{examplethree}.

Внутри каждого из этих окружений определены команды \command{exmp} и \command{exmpfile}.
В окружениях \t{example} и \t{examplewide}
они принимают два параметра "--- пример ввода и пример ответа.
Эти два окружения отличаются относительным расположением примера и ответа.
Окружение \t{examplewide} следует использовать, если пример занимает много места в ширину.

Окружение \t{examplethree} отличается наличием третьего параметра.
Его заголовок можно определить следующей командой:
\begin{lstlisting}[language=tex]
\renewcommand{\kwExampleNotes}{Your text here}
\end{lstlisting}
По умолчанию заголовок <<Пояснение>> в русских условиях и <<Note>> в английских.

Пример условия задачи можно посмотреть в \subgiturl{blob/master/problems/arithmetic/statement/arithmetic.en.tex}{репозитории}.

\newpage
\section{Настройки отображения}

Настройки различных деталий сборки условий устанавливаются в tex-файле контеста.
Этот файл содержит заголовок, в котором перечислены используемые модули и прочие 
общие для задач настройки. Пример такого файла можно посмотреть в 
\subgiturl{/blob/master/statements/121014.en.tex}{репозитории}.
Так же этот файл должен каким-либо способом подключать файлы с условиями отдельных задач.
В примере эти подключения вынесены в отдельный \subgiturl{/blob/master/statements/121014.inc.tex}{файл}.

\subsection{Название контеста}

Название контеста задается командой 
\begin{lstlisting}[language=tex]
\contest{Name}{Place}{Date}
\end{lstlisting}

В результате этой команды в шапке будет отображаться \t{Name} на первой строке,
\t{Place, Date} на второй.

\subsection{Русификсация}

Язык используемый по умолчанию "--- английский. Для того, чтобы заголовки секций и задач были на
русском языке пакету \t{olymp.sty} нужно передать опцию \t{russian}. То есть 
команда подключения пакета будет выглядить как
\begin{lstlisting}[language=tex]
\usepackage[russian]{olymp.sty}
\end{lstlisting}

\subsection{Нумерация задач}

По умолчанию задачи нумеруются заглавными английскими буквамя. Если передать пакету 
\t{olymp.sty} опцию \t{arabic}, то задачи будут нумероваться числами.

\subsection{Ориентация страницы}

\t{olymp.sty} поддерживает как вертикальную, так и горизонтальную ориентацию страницы.
Горизонтальная ориентация страницы поддерживается только в режиме с двумя колонками.
При этом каждая новая задача не будет начинаться с новой страницы.

Для того, чтобы включить горизонтальный режим, в заголовке \command{documentclass}
необходимо укзать опции \t{landscape, twocolumn}.

\subsection{Пустые страницы}
\t{olymp.sty} поддерживает вставку пустых страниц, с целью, чтобы каждая задача занимала
целое число листов, что будет удобно для двустронней печати.

Для того чтобы включить эту опцию необходимо в заголовке контеста написать команду
\command{intentionallyblankpagestrue}

\subsection{Отображение авторов}
Автор и разработчик могут указываться в двух местах "--- в заголовке задачи
рядом с ограничениями по времени или внизу страницы.

Для показа Автора или Разработчика в заголовке задачи необходимо в заголовке контеста 
написать команды 
\command{displayauthortrue} и
\command{displaydevelopertrue} соответсвенно.

Для показа Автора или Разработчика внизу страницы необходимо в заголовке контеста 
\command{displayauthorinfootertrue} и \command{displaydeveloperinfootertrue}. 
Если указаны обе команды, то будет показан только Автор.

По сложившемуся пониманию автор задачи "--- это автор идеи, разработчик "--- это
человек готовивший тесты.

При использовании этих опций, необходимо указать авторов и разработчиков задач.
Это делается командами \command{gdef}\unskip\command{thisproblemauthor\{Author\}}
и \command{gdef}\unskip\command{thisproblemdeveloper\{Developer\}} перед началом соотвествующей
задачи.

Мы рекомендуем указывать авторов и разработчиков задач в исходниках условий, даже если
они не отображаются. Ранее было принято указывать их комментариями, а не специальными командами.

\subsection{Отображение ревизии svn}

% Я не знаю как это работает, напишите кто нибудь другой

TODO





\end{document}
